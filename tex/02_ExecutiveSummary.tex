%--------------------------------------------------------------------------------------
%      EXECUTIVE SUMMARY
%--------------------------------------------------------------------------------------
\justify
A \gls{pbs} framework legislates the dynamic performance and road-width usage of heavy vehicles, allowing the length and mass of a vehicle to exceed prescriptive legislation. The PBS framework defines the safe performance envelope of vehicles but does not optimise their safety and productivity. The design process to achieve the optimal productivity of PBS vehicles is highly iterative. An initial design is evaluated using multi-body dynamics simulation. If the required PBS performance is not achieved, design iterations are made until the required PBS performance is achieved. The process is costly, time-consuming and computationally expensive. The objective of this research is to quantify the relative effect of each \gls{vdp} of a multi-body vehicle dynamics model on the vehicle safety as measured within the \gls{pbs} framework to assist in the PBS assessment process. To achieve this, three representative baseline PBS vehicles were developed (a quad semi-trailer, tridem interlink and rigid drawbar combination) from PBS assessments conducted in South Africa. A set of ranges within which each \gls{vdp} could be varied was developed by considering \gls{oem} data, legal restrictions, physical constraints and South African PBS assessments. Each \gls{vdp} for each baseline combination was varied in isolation to evaluate its influence on the vehicles performance within the \gls{pbs} framework. A comparative matrix was developed for each baseline vehicle comparing the relative influence of each \gls{vdp} on each of the PBS performance measures. The matrices yield insight into which \glspl{vdp} have the most influence on each performance measure for each of the baseline vehicles. Furthermore \glspl{vdp} that have a negligible influence on the performance of all baseline vehicles can be conservatively estimated in the absence of \gls{oem} data while still predicting representative vehicle performance. These insights will guide designers to focus on \glspl{vdp} with a high influence on vehicle performance, allow PBS assessors to determine which design parameters can be modelled with generic approximate data in the absence of \gls{oem} data, and speed up the process of assessing vehicles within the \gls{pbs} framework.