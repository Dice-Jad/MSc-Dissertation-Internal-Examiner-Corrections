\chapter{Introduction}\label{chapter:introduction}

A \glsfirst{pbs} approach to heavy vehicle design involves assessing vehicle safety performance through a set of performance measures \cite{Arredondo2012}. The vehicle is required to perform standard road manoeuvres, during which these performance measures are evaluated through either physical testing or simulation.

Benefits of the \gls{pbs} approach realised by Australia, New Zealand and Canada motivated South Africa to gain practical experience with the approach and evaluate the benefits within the South African context. The Australian \gls{pbs} framework was identified as the most suitable for South Africa and after a successful trial of two demonstration vehicles, the Australian \gls{pbs} framework was adopted \cite{Nordengen2014}. 

As of June 2017, South Africa had 245 \gls{pbs} vehicles in operation which had collectively travelled over 100~million~km within 8 of the 9 provinces. All operators participating in the \gls{pbs} pilot project are required to record monitoring data for both their legal and \gls{pbs} fleets. This data has proven that safety and productivity improvements have been realised by enforcing \gls{pbs} compliance in South Africa \cite{Nordengen2018}.

The process of assessing and optimising a heavy vehicle within the PBS framework is costly and time consuming. Initially, data for each vehicle in the combination being assessed needs to be sourced from all the relevant third-party \glspl{oem}. Should the required data be considered as proprietary by the \glsfirst{oem}, permission from overseas head offices needs to be obtained before the data may be released and a \gls{nda} may need to be signed. 

Once all of the data has been acquired, the combination is then modelled in a multi-body vehicle dynamics program to assess its performance within the \gls{pbs} framework. Should a heavy vehicle not achieve the required PBS performance level iterative modifications are made to improve the design until the required performance is achieved. 

Studies have been conducted in the past to evaluate how a selection of \glsfirst{vdp} affect vehicle performance, however they omit many of the \glspl{vdp} required to fully define a vehicle model. There is a need for a better understanding of how each \gls{vdp} influences vehicle performance within the \gls{pbs} framework. This will help assessors and designers focus on the design parameters that have a high influence on vehicle performance and spend less time tweaking parameters with a low influence.

In addition, understanding which \glspl{vdp} have an insignificant effect on vehicle performance will allow for conservative estimates to be made for these parameters without significantly degrading the accuracy of the assessment. This will help speed up \gls{pbs} assessments where \gls{oem} data is not readily supplied due to the red tape involved in distributing proprietary information which can drastically affect the time required to complete a \gls{pbs} assessment.



