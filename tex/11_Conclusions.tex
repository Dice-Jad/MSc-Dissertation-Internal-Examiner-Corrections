\chapter{Conclusions}\label{Conclusions} 

This study evaluated the relative effect of \glspl{vdp} on three of the most common \glspl{hcv}, a quad semi-trailer, tridem interlink and rigid drawbar combination. To prevent the relative influence of any \gls{vdp} being over or underestimated, a range within which each \gls{vdp} could be varied was determined by considering \gls{oem} data, legal restrictions, physical constraints and South African PBS assessment data. Evaluating the influence of each \gls{vdp} within these ranges adds insight into the limitations imposed by restrictions on the variation of a \gls{vdp} and expands on previous research conducted by Prem et al. \cite{Prem2002}.

The influence of a \gls{vdp} on vehicle performance for each \gls{pbs} performance measure was quantified with the coefficient of variation (\gls{cv}) metric. A comparative matrix (denoted as the \gls{cv} matrix) was developed for each baseline vehicle which compares the relative influence of each \gls{vdp} in terms of each of the \gls{pbs} performance measures. 

The overall \gls{cv} matrices (see Tables~\ref{table:overall-cv-quad-semi}~to~\ref{table:overall-cv-rigid}) provide insight into which \glspl{vdp} are the most influential in terms of each of the \gls{pbs} performance measures. If a proposed vehicle design fails a PBS assessment, the columns for each of the failed \gls{pbs} performance measures can be consulted to determine which \glspl{vdp} will yield the most improved performance for that \gls{pbs} performance measure.

The complete \gls{cv} matrices (see Tables~\ref{table:complete-cv-quad-semi-trailer}~to~\ref{section:complete-cv-rigid-drawbar}) highlight the \glspl{vdp} that have a negligible relative influence for all \gls{pbs} performance measures for all baseline combinations. A much larger proportion of suspension and tyre \glspl{vdp} were found to have a negligible relative influence compared to the inertial and geometric \gls{vdp}. These \glspl{vdp} (listed in Section~\ref{section:vdps-with-negligible-influence-on-overall-performance}) would likely have a negligible relative influence on all \glspl{hcv}. Using discretion, these \glspl{vdp} could be conservatively estimated and still provide realistic prediction of vehicle performance.

Additional \gls{cv} matrices were developed comparing the inertial, geometrical, suspension and tyre \glspl{vdp} in isolation (see Appendices~\ref{section:cv-geometrical}~to~\ref{section:cv-tyre}). These \gls{cv} matrices highlight the \glspl{vdp} with the most influence within each category independent of all other \glspl{vdp} and will guide efforts focussed on a specific area of vehicle design.
 
The results contained in this study will help speed up the \gls{pbs} assessment process and guide vehicle design efforts towards high-impact \glspl{vdp} when optimising vehicle design leading to safer, more productive \glspl{hcv}.